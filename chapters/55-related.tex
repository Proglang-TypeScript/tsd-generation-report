\chapter{Related Work}\label{chap:relatedwork}

\subsubsection{Microsoft's dts-gen}
Microsoft developed \mintinline{text}{dts-gen}, a tool that creates starter declaration files for JavaScript libraries \citep{dts-gen}. Its documentation states that the result is however intended to be only used a starting point. The outcome needs to be refined afterwards by the developers.

They analyze the shape of the objects at runtime after initialization without executing the library. This results in many variables being inferred as \mintinline{text}{any}. \coderef{code:related-work-dts-gen-example} shows an example for module \mintinline{text}{abs}.

The solution presented in this work, however, is intended to generate declaration files that are ready to be uploaded to DefinitelyTyped without further manual intervention. Any amount of manual work that a developer needs to do on a declaration file after updating JavaScript code increases the risk for having discrepancies between the declaration file and the implementation.

Formal aspects like applying the right template and using the correct syntax are perfectly covered by \mintinline{text}{dts-gen}.

\begin{code}
	\begin{bashinline}
$ npm i -g dts-gen
$ npm i -g abs
$ dts-gen -m abs
Wrote 5 lines to abs.d.ts.

$ cat abs.d.ts
/** Declaration file generated by dts-gen */

export = abs;

declare function abs(input: any): any;
	  \end{bashinline}
	\caption[Microsoft's dts-gen example]{\textbf{Microsoft's dts-gen example} - A declaration file for module \mintinline{text}{abs} is generated. Types are inferred as \mintinline{text}{any}. The correct \mintinline{text}{module-function} template is used.}
	\label{code:related-work-dts-gen-example}
  \end{code}


\subsubsection{TSInfer \& TSEvolve}
TSInfer and TSEvolve are presented as part of TSTools \citep{DBLP:conf/fase/KristensenM17}. Both tools are the continuation of TSCheck \citep{DBLP:conf/oopsla/FeldthausM14}, a tool for looking for mismatches between a declaration file and an implementation.

TSInfer proceeds in a similar way than TSCheck. It initializes the library in a browser and it records a snapshot of the resulting state and then it performs a light weight static analysis on all the functions and objects stored in the snapshot.

The abstraction and the constraints they introduced as part of the static analysis tools for inferring the types have room for improvement. A run-time based approach like the one presented in our work will provide more accurate information, thus generating more precise declaration files.

Since they analyze the objects and functions stored in the snapshot, they faced the problem of including in the declaration file internal methods and properties that developers wanted to hide. Run-time information would have informed that the developer has no intention of exposing such methods.

Moreover, TSEvolve performs a differential analysis on the changes made to a JavaScript library in order to determine intentional discrepancies between declaration files of two consecutive versions. We consider that a differential analysis may not be needed. If the developer's intention is accurately extracted and the execution code clearly represents that intention then the generated declaration file would already describe the newer version of a library without the need of a differential analysis.

\subsubsection{TSTest}
TSTest is a tool that checks for mismatches between a declaration file and a JavaScript implementation \citep{DBLP:journals/pacmpl/KristensenM17}. It applies feedback-directed random testing for generating type test scripts. These scripts will execute the library in order to check if it behaves the way it is described in the declaration file. TSTest also provides concrete executions for mismatches.

We evaluated the generated declaration files comparing them to the declaration files uploaded to DefinitelyTyped. The disadvantage of doing this is that since the uploaded files are written manually, they could already contain mismatches with the JavaScript implementation. However, it is a suitable choice for a development stage since it is used as a baseline.

In a final stage, declaration files need to be checked against the proper JavaScript implementation and TSTest has to be definitely taken into account.