\chapter{Introduction}\label{chap:introduction}

JavaScript has become the most popular language for writing web applications \citep{github-statistics}. It is also gaining popularity for back-end applications running in NodeJS. Its dynamic typing speeds up programming enabling developers to create simple pieces of code very fast, making JavaScript a very appealing programming language. As Avik Chaudhuri explains in his talk \textit{Type Inference for Dynamically-Typed Languages} at Microsoft \citep{type-inference-microsoft-research-video}, developers choose dynamically typed languages mainly for two reasons:

\begin{enumerate}
	\item Verbosity: By omitting type related annotations, you manage to create code that is much shorter than the equivalent code written in a strongly typed language. This is particularly interesting for trivial or easy operations, where the type related verbosity creates an overhead that sometimes obscures the developer's intention.

	\item Type related rigidity: Dynamically typed languages allow functions to return different types for different inputs. For example, a function that returns a \mintinline{text}{number} most of the time but \mintinline{text}{false} for specific inputs.
\end{enumerate}

JavaScripts allows the developer to do things that are probably not intuitive like multiplying objects and strings, adding strings and arrays or computing the minimum between a boolean and an object. On one hand, this kind of behavior helps the developer to spare several lines of code, making it more readable and understandable. On the other hand, however, JavaScript's native behavior might lead to moments of great frustration and long debugging sessions.

JavaScript is being used for creating complex and large applications. However, JavaScript was not intended to be more than a scripting language. Maintaining and evolving large JavaScript codebases is notably challenging. Mistakes such as mistyped property names and misunderstood or unexpected type coercion cause developers to spend a significant amount of time in debugging sessions. A JavaScript code blog\footnote{https://wtfjs.com} compiles experiences from developers facing unexpected situations while programming in JavaScript. \coderef{code:introduction-javascript-wtfs} exposes some of these unintuitive JavaScript behaviors.

\begin{code}
	\jscode{code/introduction/javascript-wtfs.js}
	\captionsetup{aboveskip=0pt, belowskip=10pt}
	\caption[Unexpected JavaScript behavior]{\textbf{Unexpected JavaScript behavior} - Examples about falsy values, \mintinline{text}{typeof} operator, \mintinline{text}{null} vs \mintinline{text}{undefined} and type coercion.}
	\label{code:introduction-javascript-wtfs}
  \end{code}

The overhead produced by such dynamic typing is not present in other languages that use build tools based on type information. The situation motivated the creation of TypeScript, a superset of JavaScript with typed annotations \citep{typescript}. It has become a widely used alternative among JavaScript developers, since it incorporates features that are helpful for developing and maintaining large applications \citep{DBLP:conf/icse/GaoBB17}. TypeScript enables the early detection of run-time errors detection and the integration of code intelligence tools like auto-completion in the IDEs.

Existing JavaScript libraries can be used in a TypeScript project by adding a declaration file that contains a typed description of the library's API. Declaration files are stored in a repository called DefinitelyTyped that contains declaration files for more than 6000 JavaScript libraries \citep{definitely-typed-repository}. Unfortunately, declaration files need to be manually created and maintained, which is error prone and time consuming. TypeScript does not perform any run-time check on these declaration files. A discrepancy between the declaration file and its corresponding JavaScript library would lead to additional frustration and debugging sessions, since type checks and code-intelligence features would be inaccurate.

Some previous work tackled the problem of automatically searching for mismatches between a declaration file and implementation code \citep{DBLP:conf/oopsla/FeldthausM14}. TSTest adapted the feedback-directed random testing technique, mainly used for testing Java libraries, to detect discrepancies between a declaration file and a JavaScript library \citep{DBLP:journals/pacmpl/KristensenM17}. Tools like TSInfer and TSEvolve are designed for assisting the creation of new declaration files and supporting the evolution the declaration file when the corresponding JavaScript library gets modified \citep{DBLP:conf/fase/KristensenM17}, respectively. TypeScript itself developed \mintinline{text}{dts-gen}, a tool that generates a declaration file that is meant to be used only as a `starting point for writing a high-quality declaration file' \citep{dts-gen}.

We explore in this work the possibilities for improving the existing tools. We present the tool \mintinline{text}{dts-generate}. It is based on an architecture that supports the generation of declaration files for an existing JavaScript library published to the NPM registry. The tool will gather data flow and type information at run-time and generate a declaration file based on that information.

The architecture supports the future incorporation of a Symbolic Execution Engine that expands the initial code base using the signatures in the declaration file. The iterative process of exploring new execution paths will refine the generated declaration file in each iteration.

TypeInference for JavaScript is studied based on favoring patterns commonly used among developers. We provide evidence of how JavaScript operators are used by analyzing more than 400 libraries.

Finally, we generated declaration files for 244 JavaScript libraries and evaluated them against DefinitelyTyped.

\subsubsection{Artifacts}
The implementation and the results can be found in \url{https://bitbucket.org/dtsgenerate}. Each repository contains a readme file that explains how to build and use the tools.

\begin{itemize}
	\item Run-time information gathering: \url{https://bitbucket.org/dtsgenerate/run-time-information-gathering}
	\item Generation of declaration files: \url{https://bitbucket.org/dtsgenerate/ts-declaration-file-generator}
	\item Service for locally generating declaration files: \url{https://bitbucket.org/dtsgenerate/ts-declaration-file-generator-service}
	\item \mintinline{text}{dts-generate} executable: \url{https://bitbucket.org/dtsgenerate/dts-generate-method}
	\item A study on JavaScript type coercion: \url{https://github.com/f-cristiani/js-type-coercion}
	\item Results: \url{https://bitbucket.org/dtsgenerate/dts-generate-results}
	\item This document: \url{https://bitbucket.org/dtsgenerate/tsd-generation-report}
\end{itemize}